\documentclass[11pt, a4paper]{article}
\usepackage[english]{babel}
\usepackage[utf8]{inputenc}
\usepackage{fancyhdr}
\usepackage{lastpage}
\usepackage{datetime}
\usepackage{indentfirst}
\usepackage{hyperref}
\usepackage{appendix}
\usepackage{amsmath}
\usepackage{amssymb}
\usepackage{amsfonts}
\usepackage{mathtools}
\usepackage{siunitx}
\usepackage{cancel}
\usepackage{tabularray}
\usepackage{multirow}
\usepackage{array}
\usepackage{hhline}
\usepackage{makecell}
\usepackage{courier}
\usepackage[font=small, skip=0pt]{caption}
\usepackage[font=scriptsize, skip=0pt]{subcaption}
\usepackage{float}
\usepackage{graphicx}
\usepackage{listings}
\usepackage{xcolor}
\usepackage{matlab-prettifier}
\usepackage[T1]{fontenc}
\usepackage{lmodern}
\usepackage{bigfoot}
\usepackage{filecontents}
\usepackage[nottoc]{tocbibind}

\graphicspath{ {./mathimages/} }

\newdateformat{Datea}{\THEDAY\ \monthname[\THEMONTH] \THEYEAR}
\newdateformat{Dateb}{\monthname[\THEMONTH] \THEYEAR}

%\allowdisplaybreaks
\DeclareMathOperator{\cosec}{cosec}
\DeclareMathOperator{\cotan}{cotan}
\DeclareMathOperator{\sech}{sech}
\DeclareMathOperator{\cosech}{cosech}
\DeclareMathOperator{\arcsec}{arcsec}
\DeclareMathOperator{\arccot}{arccot}
\DeclareMathOperator{\arccsc}{arccosec}
\DeclareMathOperator{\arccosh}{arccosh}
\DeclareMathOperator{\arcsinh}{arcsinh}
\DeclareMathOperator{\arctanh}{arctanh}
\DeclareMathOperator{\arcsech}{arcsech}
\DeclareMathOperator{\arccsch}{arccsch}
\DeclareMathOperator{\arccoth}{arccoth}
\DeclareMathOperator{\arsinh}{arsinh}
\DeclareMathOperator{\arcosh}{arcosh}
\DeclareMathOperator{\artanh}{artanh}

\DeclareMathOperator{\cis}{cis}

\pagestyle{fancy}
\fancyhf{}
\rhead{Hatam Barma}
\chead{\begin{tabular}[t]{@{}l@{}}\\Mathematics and Further Mathematics Pure Revision Summary\end{tabular}}
\lhead{\Dateb\today}
\cfoot{Page \thepage}

\renewcommand{\thesection}{\arabic{section}} 

\renewcommand{\thesubsection}{\thesection.\arabic{subsection}}

\setcounter{section}{8}

\allowdisplaybreaks

\fancypagestyle{plain}{
\fancyhf{}
\renewcommand{\headrulewidth}{0pt}}

\hypersetup{
    colorlinks,
    citecolor=black,
    filecolor=black,
    linkcolor=blue,
    urlcolor=magenta!70!black
}

\begin{document}


\begin{titlepage}
   \begin{center}
       \vspace*{2.5cm}
	\huge
       \textbf{A-Level Mathematics and Further Mathematics Pure Revision Summary} \\
	\vspace{1cm}
	\Large
       \textbf{Chapter 9: Partial fractions}
            
       \vspace{1.5cm}
	\LARGE
       \textbf{Hatam Barma} \\
	\vspace{0.75cm}
       \normalsize
       \emph{Compiled on \Datea\today} \\

       \vfill
        

	E-mail: hatam.barma@gmail.com
   \end{center}
\end{titlepage}


\tableofcontents

\clearpage
\section{Partial fractions}
\vspace{0.5cm}

\subsection{Partial fractions}
\label{partialfractions1}
\begin{itemize}
\item A Level M Year 2 \hspace{1cm} \phantom{ AS / } Pages 99 -- 104
\item A Level M Year 2 \hspace{1cm} \phantom{ AS / } Pages 239 -- 243
\end{itemize} \par
Partial fractions is a way of decomposing rational polynomials. Here are a couple of examples;
\scriptsize
\begin{align*}
\frac{1}{(2x-3)}-\frac{3}{2(x+1)}&=\frac{-4x+11}{2(x+1)(2x-3)} & \frac{\text{Linear}}{\text{Quadratic (distinct factors)}} \\
\frac{2}{(x-2)}+\frac{5}{(x-2)^{2}}&=\frac{2x+1}{(x-2)^{2}} & \frac{\text{Linear}}{\text{Quadratic (repeated linear factor)}} \\
\frac{1}{(x)}-\frac{3}{(x)^{2}}+\frac{1}{(3x+5)}&=\frac{4x^{2}-4x-15}{(x)^{2}(3x+5)} & \frac{\text{Quadratic}}{\text{Cubic (one repeated linear factor, one other)}}
\end{align*}
\normalsize

To utilise partial fractions, the degree of the numerator must be less than the degree of the denominator. So, to change the degree of the numerator, take out a factor of the denominator from the numerator, and then split the remainder into partial fractions.

\begin{flalign*}
\frac{x^{3}-5x^{2}+2x+14}{(x-3)^{2}}&\equiv\frac{(x-3)^{2}(x+1)-x+5}{(x-3)^{2}} && \\
&\equiv \frac{\cancel{(x-3)^{2}}(x+1)}{\cancel{(x-3)^{2}}}+\frac{-x+5}{(x-3)^{2}} && \\
&\equiv (x+1)+\frac{-x+5}{(x-3)^{2}}
\end{flalign*}
\begin{flalign*}
\frac{-x+5}{(x-3)^{2}}&=\frac{A}{(x-3)}+\frac{B}{(x-3)^{2}} && \\
&=\frac{A(x-3)+B}{(x-3)^{2}} \hspace{2cm} \Rightarrow A(x-3)+B=-x+5
\end{flalign*}
From here, either equate coefficients or substitute in for $x$ (use $x=3$ in this case to eliminate A) $\Rightarrow A=-1$; $B=2$. Therefore;
\begin{equation*}
\frac{x^{3}-5x^{2}+2x+14}{(x-3)^{2}}\equiv(x+1)-\frac{1}{(x-3)}+\frac{2}{(x-3)^{2}}
\end{equation*}
Partial fractions can be used to integrate functions of rational polynomials more easily.
\vspace{0.5cm}


\subsection{Forms of partial fractions}
\label{partialfractions2}
\begin{itemize}
\item A Level M Year 2 \hspace{1cm} \phantom{ AS / } Pages 99 -- 104
\item A Level M Year 2 \hspace{1cm} \phantom{ AS / } Pages 239 -- 243
\end{itemize} \par
\begin{centering}
\begin{tblr}{|[.75pt]|l|c|c||[.75pt]}
\hline[1.25pt]
Factor of denominator & Example & Partial fraction to include \\ \hline[.75pt]
Distinct linear & $(2x+5)$ & $\frac{A}{(2x+5)}$ \\ \hline
Repeated linear & $(2x+7)^{3}$ & $\frac{A}{(2x+7)^{3}}+\frac{B}{(2x+7)^{2}}+\frac{C}{(2x+7)}$ \\ \hline
\SetCell[r=3]{c} Irreducible Quadratic & $(x^{2}+4)$ & $\frac{Ax+B}{(x^{2}+4)}$  \\
& $(x^{2}+x+1)$ & $\frac{Ax+B}{x^{2}+x+1}$ \\
& $(x^{2}+4x+1)$ & $\frac{Ax+B}{x^{2}+4x+1}$ \\ 
\hline[.75pt]
\end{tblr}
\end{centering} \newline \par
\scriptsize We would choose not to factorise the irreducible quadratics in order to avoid complex roots \normalsize
\vspace{0.5cm}

\end{document}