\documentclass[11pt, a4paper]{article}
\usepackage[english]{babel}
\usepackage[utf8]{inputenc}
\usepackage{fancyhdr}
\usepackage{lastpage}
\usepackage{datetime}
\usepackage{indentfirst}
\usepackage{hyperref}
\usepackage{appendix}
\usepackage{amsmath}
\usepackage{amssymb}
\usepackage{amsfonts}
\usepackage{mathtools}
\usepackage{siunitx}
\usepackage{cancel}
\usepackage{tabularray}
\usepackage{multirow}
\usepackage{array}
\usepackage{hhline}
\usepackage{makecell}
\usepackage{courier}
\usepackage[font=small, skip=0pt]{caption}
\usepackage[font=scriptsize, skip=0pt]{subcaption}
\usepackage{float}
\usepackage{graphicx}
\usepackage{listings}
\usepackage{xcolor}
\usepackage{matlab-prettifier}
\usepackage[T1]{fontenc}
\usepackage{lmodern}
\usepackage{bigfoot}
\usepackage{filecontents}
\usepackage[nottoc]{tocbibind}

\graphicspath{ {./mathimages/} }

\newdateformat{Datea}{\THEDAY\ \monthname[\THEMONTH] \THEYEAR}
\newdateformat{Dateb}{\monthname[\THEMONTH] \THEYEAR}

%\allowdisplaybreaks
\DeclareMathOperator{\cosec}{cosec}
\DeclareMathOperator{\cotan}{cotan}
\DeclareMathOperator{\sech}{sech}
\DeclareMathOperator{\cosech}{cosech}
\DeclareMathOperator{\arcsec}{arcsec}
\DeclareMathOperator{\arccot}{arccot}
\DeclareMathOperator{\arccsc}{arccosec}
\DeclareMathOperator{\arccosh}{arccosh}
\DeclareMathOperator{\arcsinh}{arcsinh}
\DeclareMathOperator{\arctanh}{arctanh}
\DeclareMathOperator{\arcsech}{arcsech}
\DeclareMathOperator{\arccsch}{arccsch}
\DeclareMathOperator{\arccoth}{arccoth}
\DeclareMathOperator{\arsinh}{arsinh}
\DeclareMathOperator{\arcosh}{arcosh}
\DeclareMathOperator{\artanh}{artanh}

\DeclareMathOperator{\cis}{cis}

\pagestyle{fancy}
\fancyhf{}
\rhead{Hatam Barma}
\chead{\begin{tabular}[t]{@{}l@{}}\\Mathematics and Further Mathematics Pure Revision Summary\end{tabular}}
\lhead{\Dateb\today}
\cfoot{Page \thepage}

\renewcommand{\thesection}{\arabic{section}} 

\renewcommand{\thesubsection}{\thesection.\arabic{subsection}}

\setcounter{section}{5}

\allowdisplaybreaks

\fancypagestyle{plain}{
\fancyhf{}
\renewcommand{\headrulewidth}{0pt}}

\hypersetup{
    colorlinks,
    citecolor=black,
    filecolor=black,
    linkcolor=blue,
    urlcolor=magenta!70!black
}

\begin{document}


\begin{titlepage}
   \begin{center}
       \vspace*{2.5cm}
	\huge
       \textbf{A-Level Mathematics and Further Mathematics Pure Revision Summary} \\
	\vspace{1cm}
	\Large
       \textbf{Chapter 6: Exponents, logarithms, hyperbolic functions}
            
       \vspace{1.5cm}
	\LARGE
       \textbf{Hatam Barma} \\
	\vspace{0.75cm}
       \normalsize
       \emph{Compiled on \Datea\today} \\

       \vfill
        

	E-mail: hatam.barma@gmail.com
   \end{center}
\end{titlepage}


\tableofcontents


\clearpage
\section{Exponents, logarithms, hyperbolic functions}
\vspace{0.5cm}

\subsection{Laws of indices}
\begin{itemize}
\item A Level M AS / Year 1 \hspace{1cm} \phantom{ } Pages 17 -- 21
\end{itemize} \par
\begin{itemize}
\item[-] Given the expression:
\begin{equation*}
a^{n}
\end{equation*}
$a$ is the base, and $n$ is the index / exponent
\item[-]A negative index gives a reciprocal
\begin{equation*}
a^{-n}=\frac{1}{a^{n}}=\left(\frac{1}{a}\right)^{n}
\end{equation*}
\item[-] Fractions in the exponent correspond to surds / roots of the base
\begin{equation*}
x^{\frac{a}{b}}=\sqrt[b]{x^{a}}=\left(\sqrt[b]{x}\right)^{a}
\end{equation*}
\end{itemize}
\vspace{0.5cm}


\subsection{Laws of logarithms}
\begin{itemize}
\item A Level M AS / Year 1 \hspace{1cm} \phantom{ } Pages 114 -- 126
\end{itemize}
\begin{flalign*}
\log_{a}(x)&=b \text{ means } a^{b}=x & \log_{a}\left(x^{n}\right)&=n\log_{a}(x)&& \\
\log_{a}\left(a^{x}\right)&=x & \log_{a}(xy)&=\log_{a}(x)+\log_{a}(y) && \\
\log_{a}(1)&=0 & \log_{a}\left(\frac{x}{y}\right)&=\log_{a}(x)-\log_{a}(y) &&
\end{flalign*}
The last two rules, for $\log_{a}(xy)$ and $\log_{a}\left(\frac{x}{y}\right)$ can be proved fairly simply. The proof for $\log_{a}(xy)$ is detailed below, and the proof for $\log_{a}\left(\frac{x}{y}\right)$ follows a very similar method.
\begin{equation*}
xy=\left( a^{\log_{a}(x)} \right)\left( a^{\log_{a}(y)} \right) = a^{\log_{a}(x)+\log_{a}(y)} = a^{\log_{a}(xy)}
\end{equation*}

\newpage


\subsection{Euler's number, `$e$'}
\begin{itemize}
\item A Level M Year 2 \hspace{1cm} \phantom{ AS / } Pages 185 -- 195
\end{itemize} \par
The number $e$ is defined such that
\begin{equation*}
f(x)=f'(x)
\end{equation*}
This is called the exponential function;
\begin{equation*}
f(x)=e^{x}=\exp(x)
\end{equation*}

From first principles,
\begin{flalign*}
\frac{\mathrm{d}}{\mathrm{d}x}[f(x)]&=\lim_{h \to 0} \left[ \frac{f(x+h)-f(x)}{h} \right] && \\
&=\lim_{h \to 0} \left[ \frac{e^{x+h}-e^{x}}{h} \right] && \\
&=\lim_{h \to 0} \left[ \frac{e^{x}\left(e^{h}-1\right)}{h} \right] && \\
&=e^{x}\lim_{h \to 0} \left[ \frac{\left(e^{h}-1\right)}{h} \right] \hspace{1cm} \text{and }\lim_{h \to 0} \left[ \frac{\left(e^{h}-1\right)}{h} \right]=1&& \\
&=e^{x}\times 1 && \\
&=e^{x} \hspace{1cm} \text{ as required}
\end{flalign*}
The logarithm with base $e$ is commonly denoted as
\begin{equation*}
\log_{e}(x)=\ln(x)
\end{equation*}
and is called the \emph{natural logarithm}
\begin{itemize}
\item[Note:] The chain rule still applies when differentiating, therefore:
\begin{equation*}
\frac{\mathrm{d}}{\mathrm{d}x}\left[ e^{f(x)} \right] = e^{f(x)} \times \frac{\mathrm{d}}{\mathrm{d}x}\left[ f(x) \right]
\end{equation*}
\end{itemize}
\vspace{0.5cm}


\subsection{Exponential models}
\begin{itemize}
\item A Level M AS / Year 1 \hspace{1cm} \phantom{ } Pages 128 -- 145
\end{itemize} \par
Exponential models are characterised by the following properties;
\begin{itemize}
\item[-] Where $A$ and $b$ are constants;\begin{equation*}y=Ae^{bx}\end{equation*}
\item[-] Rate of change is proportional to the remaining / current quantity
\begin{equation*}\frac{\mathrm{d}y}{\mathrm{d}x}=b\cdot Ae^{bx}=b\cdot y \end{equation*}
\item[-] Fixed changes in the $x$ value correspond with fixed \emph{scale factor} changes in the output
\item[-]Exponential models can be linearised by taking natural logarithms of both sides
\begin{align*}
y&=Ae^{bx} \\
\ln(y)&=\ln\left(Ae^{bx}\right) \\
&=\ln(A)+\ln\left(e^{bx}\right) \\
&=\ln(A)+b\ln(e^{x}) \\
&=\ln(A)+bx
\end{align*}
\end{itemize}
\vspace{0.5cm}


\subsection{The derivative of ln(x)}
\begin{itemize}
\item A Level M Year 2 \hspace{1cm} \phantom{ AS / } Pages 214 -- 216
\end{itemize} \par
The derivative of $\ln(x)$ is
\begin{equation*}
\frac{\mathrm{d}}{\mathrm{d}x}\left[ \ln(x) \right]=\frac{1}{x}
\end{equation*}
and so the integral of $\frac{1}{x}$ is 
\begin{equation*}
\int\frac{1}{x}=\ln(x)+c
\end{equation*}
Again, this can be used in conjunction with the chain rule. A couple of examples follow:
\begin{align*}
\frac{\mathrm{d}}{\mathrm{d}x}\left[ \ln(something) \right]&=\frac{1}{(something)}\times\frac{\mathrm{d}}{\mathrm{d}x}\left[ something \right] \\
\frac{\mathrm{d}}{\mathrm{d}x}\left[ \ln\left(\sin(x)\right) \right]&=\frac{1}{\sin(x)}\times\cos(x)=\cot(x) \\
\frac{\mathrm{d}}{\mathrm{d}x}\left[ \sin\left(\ln(x)\right) \right]&=\cos\left(\ln(x)\right)\times\frac{1}{x}=\frac{\cos\left(\ln(x)\right)}{x} \\
\end{align*}


\subsection{Expanding the domain -- ln(|x|)}
The function $y=\ln(x)$ only is valid for inputs of $x>0$. However, the curve $y=\frac{1}{x}$ is valid for $x\in\mathbb{R}\setminus{0}$. Therefore to expand the validity of $ln(x)$ as a solution to the integral of $\frac{1}{x}$, we must use the modulus of $x$ inside the argument of the natural logarithm.
\begin{align*}
\frac{\mathrm{d}}{\mathrm{d}x}\left[ \ln(x) \right]&=\frac{1}{x} \, \, \text{for } x>0 \\
\frac{\mathrm{d}}{\mathrm{d}x}\left[ \ln\left(\textcolor{red}{|}x\textcolor{red}{|}\right) \right]&=\frac{1}{x} \, \, \text{for } x\in\mathbb{R}\setminus{0} \\
\end{align*}


\subsection{The hyperbolic functions}
\begin{itemize}
\item A Level FM Year 2 \hspace{1cm} \phantom{AS /} Pages 129 -- 131
\item A Level FM Year 2 \hspace{1cm} \phantom{AS /} Pages 138 -- 142
\end{itemize} \par
The \emph{hyperbolic} functions are defined as follows:
\begin{gather*}
\cosh(x)=\frac{e^{x}+e^{-x}}{2} \\
\sinh(x)=\frac{e^{x}-e^{-x}}{2} \\
\tanh(x)=\frac{\sinh(x)}{\cosh(x)}=\frac{e^{x}-e^{-x}}{e^{x}+e^{-x}}
\end{gather*}
Each with domain $\mathbb{R}$ \newline \par

The \emph{inverse hyperbolic} functions are defined as follows:
\begin{align*}
\sech(x)&=\frac{1}{\cosh(x)}=\frac{e^{x}+e^{-x}}{2} \\
\cosech(x)&=\frac{1}{\sinh(x)}=\frac{e^{x}-e^{-x}}{2} \\
\coth(x)&=\frac{\cosh(x)}{\sinh(x)}=\frac{e^{x}+e^{-x}}{e^{x}-e^{-x}}
\end{align*}
Each with domain $\mathbb{R}$
\vspace{0.5cm}

\newpage

\subsection{Calculus of the hyperbolic functions}
\begin{itemize}
\item A Level FM Year 2 \hspace{1cm} \phantom{AS /} Pages 142 -- 149
\end{itemize} \par
\begin{flalign*}
\frac{\mathrm{d}}{\mathrm{d}x}\left[ \cosh(x) \right] &= \frac{\mathrm{d}}{\mathrm{d}x}\left[ \frac{e^{x}}{2}+\frac{e^{-x}}{2} \right]=\frac{e^{x}}{2}-\frac{e^{-x}}{2}=\frac{1}{2}\left(e^{x}-e^{-x}\right)=\sinh(x) && \\
\frac{\mathrm{d}}{\mathrm{d}x}\left[ \sinh(x) \right] &= \frac{\mathrm{d}}{\mathrm{d}x}\left[ \frac{e^{x}}{2}-\frac{e^{-x}}{2} \right]=\frac{e^{x}}{2}+\frac{e^{-x}}{2}=\frac{1}{2}\left(e^{x}+e^{-x}\right)=\cosh(x) && \\
\frac{\mathrm{d}}{\mathrm{d}x}\left[ \tanh(x) \right] &=\frac{\mathrm{d}}{\mathrm{d}x}\left[ \frac{e^{x}-e^{-x}}{e^{x}-e^{-x}} \right] && \\
&=\frac{\left( e^{x}+e^{-x} \right)\left( e^{x}+e^{-x} \right)-\left( e^{x}-e^{-x} \right)\left( e^{x}+e^{-x} \right)}{\left( e^{x}+e^{-x} \right)^{2}} && \\
&=\frac{\left( e^{x}+e^{-x} \right)^{2}-\left( e^{x}-e^{-x} \right)^{2}}{\left( e^{x}+e^{-x} \right)^{2}} && \\
&=\frac{\cancel{e^{2x}}+\cancel{e^{-2x}}+2-\cancel{e^{2x}}-\cancel{e^{-2x}}+2}{\left( e^{x}+e^{-x} \right)^{2}} && \\
&=\left( \frac{4}{\left( e^{x}+e^{-x} \right)^{2}} \right) = \left( \frac{2}{e^{x}+e^{-x}} \right)^{2}=\sech^{2}(x)&& \\
\end{flalign*}


\subsection{The inverse hyperbolic functions}
\begin{itemize}
\item A Level FM Year 2 \hspace{1cm} \phantom{AS /} Pages 131 -- 135
\end{itemize} \par
The inverse hyperbolic functions can be found by defining $y$ equal to the inverse function, rearranging to give $x$ in terms of $y$, so that $x$ is equal to the original hyperbolic, and then using the explicit form in exponentials to rearrange for $y$ in terms of $x$. This is demonstrated for the three hyperbolic functions below. \newline \par

\subsubsection*{arcosh}
\vspace{-0.8cm}
\begin{flalign*}
y&=\arcosh(x) && \\
x&=\cosh(y) && \\
x&=\frac{e^{y}+e^{-y}}{2} && \\
2x&=e^{y}+e^{-y} && \\
2x\cdot e^{y}&=e^{2y}+1 && \\
0&=\left( e^{y} \right)^{2}-2x\cdot e^{y}+1 \hspace{1cm} \text{This is simply a quadratic in $e^{y}$} && \\
e^{y}&=\frac{2x\pm\sqrt{4x^{2}-4}}{2} && \\
e^{y}&=x\pm\sqrt{x^{2}-1} && \\
y&=\ln\left( x+\sqrt{x^{2}-1} \right)
\end{flalign*}
Choose positive as convention, though both solutions satisfy the equation \newline \par

\subsubsection*{arsinh}
\vspace{-0.8cm}
\begin{flalign*}
y&=\arsinh(x) && \\
x&=\sinh(y) && \\
x&=\frac{e^{y}-e^{-y}}{2} && \\
2x&=e^{y}-e^{-y} && \\
2x\cdot e^{y}&=e^{2y}-1 && \\
0&=\left( e^{y} \right)^{2}-2x\cdot e^{y}-1 \hspace{1cm} \text{This is simply a quadratic in $e^{y}$} && \\
e^{y}&=\frac{2x\pm\sqrt{4x^{2}+4}}{2} && \\
e^{y}&=x\pm\sqrt{x^{2}+1} && \\
e^{y}&=x\textcolor{red}{+}\sqrt{x^{2}+1} && \\
y&=\ln\left( x+\sqrt{x^{2}+1} \right)
\end{flalign*}
We reject the $x-\sqrt{x^{2}+1}$ root, because $e^{y}>0$ and since $x^{2}+1>x^{2}$, $\sqrt{x^{2}+1}>x$, so $x-\sqrt{x^{2}+1}<0$, and so is not a valid solution. \newline \par

\subsubsection*{artanh}
\vspace{-0.8cm}
\begin{flalign*}
y=\artanh(x)& && \\
x=\tanh(y)=\frac{\sinh(x)}{\cosh(x)}&=\frac{e^{y}-e^{-y}}{e^{y}+e^{-y}}=\frac{e^{2y}-1}{e^{2y}+1} && \\
\left( e^{2y}+1 \right)x&=xe^{2y}+x=e^{2y}-1 && \\
xe^{2y}-e^{2y}&=-x-1 && \\
e^{2y}(x-1)&=(-x-1) && \\
e^{2y}&=\frac{1+x}{1-x} && \\
y&=\frac{1}{2}\ln\left(\frac{1+x}{1-x}\right)
\end{flalign*}

\subsection{Hyperbolic identities}
\begin{itemize}
\item A Level FM Year 2 \hspace{1cm} \phantom{AS /} Pages 136 -- 141
\end{itemize} \par
There are many identities involving hyperbolic functions, all of which are similar to trigonometric identities. The relationship between hyperbolic and trigonometric identities is covered by Osborne's Rule (section \ref{osborne}), but for now, here are the important ones to learn.
\begin{align*}
\cosh(x)+\sinh(x)&= e^{x} & \cosh(x)-\sinh(x)&= e^{-x} \\
\cosh^{2}(x)-\sinh^{2}(x)&= 1 & 1-\tanh^{2}(x)&=\sech^{2}(x) \\
\coth^{2}(x)-1&=\cosech^{2}(x) & \cosh^{2}+\sinh^{2}(x)&=\cosh(2x) \\
2\cosh^{2}(x)+1&=\cosh(2x) & 1+2\sinh^{2}(x)&=\cosh(2x) \\
\frac{2\tanh(x)}{1+\tanh^{2}(x)}&=\tanh(2x) & 2\sinh(x)\cosh(x)&=\sinh(2x)
\end{align*}

\begin{align*}
\cosh(x+y)&=\cosh(x)\cosh(y)+\sinh(x)\sinh(y) \\
\cosh(x)+\cosh(y)&=2\cosh\left( \frac{x+y}{2} \right)\cosh\left( \frac{x-y}{2} \right) \\
\end{align*}


\subsection{Osborne's Rule}
\label{osborne}
Osborne's Rule is a way of taking trigonometric identities, and obtaining its equivalent hyperbolic identity. See here the similarities between some trigonometric and hyperbolic identities;
\begin{align*}
\cos^{2}(x)+\sin^{2}(x)&=1 & \cosh^{2}(x)-\sinh^{2}(x)&=1 \\
\cos^{2}(x)-\sin^{2}(x)&=\cos(2x) & \cosh^{2}(x)+\sinh^{2}(x)&=\cosh(2x) \\
2\sin(x)\cos(x)&=\sin(2x) & 2\sinh(x)\cosh(x)&=\sinh(2x)
\end{align*}
Osborne's Rule:
\begin{enumerate}
\item Take a trigonometric identity involving terms of sine and cosine \textbf{\underline{only}}
\item Replace any $\sin$ terms with $\sinh$, and any $\cos$ terms with $\cosh$
\item Change the sign in front of any terms involving $\sinh^{2}$
\end{enumerate}
\vspace{0.5cm}


\subsection{Derivatives of the inverse hyperbolic functions}
\begin{itemize}
\item A Level FM Year 2 \hspace{1cm} \phantom{AS /} Pages 155 -- 162
\end{itemize} \par
Derivatives of the inverse hyperbolic functions can be found using implicit differentiation (see section \ref{implicitdifferentiation}). This is the origin of some of the standard integrals listed in section \ref{usefulintegrals}

\subsubsection*{arcosh}
\vspace{-0.8cm}
\begin{flalign*}
y&=\arcosh(x) && \\
x&=\cosh(y) && \\
1&=\sinh(y)\frac{\mathrm{d}y}{\mathrm{d}x} && \\
\frac{\mathrm{d}y}{\mathrm{d}x}&=\frac{1}{\sinh(y)}=\frac{1}{\pm\sqrt{\cosh^{2}(y)-1}}=\frac{1}{\pm\sqrt{x^{2}-1}}
\end{flalign*}
\begin{equation*}
\frac{\mathrm{d}y}{\mathrm{d}x}\left[\arcosh(x)\right]=\frac{1}{\sqrt{x^{2}-1}}
\end{equation*}

\subsubsection*{arsinh}
\vspace{-0.8cm}
\begin{flalign*}
y&=\arsinh(x) && \\
x&=\sinh(y) && \\
1&=\cosh(y)\frac{\mathrm{d}y}{\mathrm{d}x} && \\
\frac{\mathrm{d}y}{\mathrm{d}x}&=\frac{1}{\cosh(y)}=\frac{1}{\pm\sqrt{\sinh^{2}(y)+1}}=\frac{1}{\pm\sqrt{x^{2}+1}}
\end{flalign*}
\begin{equation*}
\frac{\mathrm{d}y}{\mathrm{d}x}\left[\arsinh(x)\right]=\frac{1}{\sqrt{x^{2}+1}}
\end{equation*}

\subsubsection*{artanh}
\vspace{-0.8cm}
\begin{flalign*}
y&=\artanh(x) && \\
x&=\tanh(y) && \\
1&=\sech^{2}(y)\frac{\mathrm{d}y}{\mathrm{d}x} && \\
\frac{\mathrm{d}y}{\mathrm{d}x}&=\frac{1}{\sech^{2}(y)}=\frac{1}{1-\tanh^{2}(y)}=\frac{1}{1-x^{2}}
\end{flalign*}
\begin{equation*}
\frac{\mathrm{d}y}{\mathrm{d}x}\left[\artanh(x)\right]=\frac{1}{1-x^{2}}
\end{equation*}
\vspace{0.5cm}


\end{document}