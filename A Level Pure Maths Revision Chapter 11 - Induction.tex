\documentclass[11pt, a4paper]{article}
\usepackage[english]{babel}
\usepackage[utf8]{inputenc}
\usepackage{fancyhdr}
\usepackage{lastpage}
\usepackage{datetime}
\usepackage{indentfirst}
\usepackage{hyperref}
\usepackage{appendix}
\usepackage{amsmath}
\usepackage{amssymb}
\usepackage{amsfonts}
\usepackage{mathtools}
\usepackage{siunitx}
\usepackage{cancel}
\usepackage{tabularray}
\usepackage{multirow}
\usepackage{array}
\usepackage{hhline}
\usepackage{makecell}
\usepackage{courier}
\usepackage[font=small, skip=0pt]{caption}
\usepackage[font=scriptsize, skip=0pt]{subcaption}
\usepackage{float}
\usepackage{graphicx}
\usepackage{listings}
\usepackage{xcolor}
\usepackage{matlab-prettifier}
\usepackage[T1]{fontenc}
\usepackage{lmodern}
\usepackage{bigfoot}
\usepackage{filecontents}
\usepackage[nottoc]{tocbibind}

\graphicspath{ {./mathimages/} }

\newdateformat{Datea}{\THEDAY\ \monthname[\THEMONTH] \THEYEAR}
\newdateformat{Dateb}{\monthname[\THEMONTH] \THEYEAR}

%\allowdisplaybreaks
\DeclareMathOperator{\cosec}{cosec}
\DeclareMathOperator{\cotan}{cotan}
\DeclareMathOperator{\sech}{sech}
\DeclareMathOperator{\cosech}{cosech}
\DeclareMathOperator{\arcsec}{arcsec}
\DeclareMathOperator{\arccot}{arccot}
\DeclareMathOperator{\arccsc}{arccosec}
\DeclareMathOperator{\arccosh}{arccosh}
\DeclareMathOperator{\arcsinh}{arcsinh}
\DeclareMathOperator{\arctanh}{arctanh}
\DeclareMathOperator{\arcsech}{arcsech}
\DeclareMathOperator{\arccsch}{arccsch}
\DeclareMathOperator{\arccoth}{arccoth}
\DeclareMathOperator{\arsinh}{arsinh}
\DeclareMathOperator{\arcosh}{arcosh}
\DeclareMathOperator{\artanh}{artanh}

\DeclareMathOperator{\cis}{cis}

\pagestyle{fancy}
\fancyhf{}
\rhead{Hatam Barma}
\chead{\begin{tabular}[t]{@{}l@{}}\\Mathematics and Further Mathematics Pure Revision Summary\end{tabular}}
\lhead{\Dateb\today}
\cfoot{Page \thepage}

\renewcommand{\thesection}{\arabic{section}} 

\renewcommand{\thesubsection}{\thesection.\arabic{subsection}}

\setcounter{section}{10}

\allowdisplaybreaks

\fancypagestyle{plain}{
\fancyhf{}
\renewcommand{\headrulewidth}{0pt}}

\hypersetup{
    colorlinks,
    citecolor=black,
    filecolor=black,
    linkcolor=blue,
    urlcolor=magenta!70!black
}

\begin{document}


\begin{titlepage}
   \begin{center}
       \vspace*{2.5cm}
	\huge
       \textbf{A-Level Mathematics and Further Mathematics Pure Revision Summary} \\
	\vspace{1cm}
	\Large
       \textbf{Chapter 11: Induction}
            
       \vspace{1.5cm}
	\LARGE
       \textbf{Hatam Barma} \\
	\vspace{0.75cm}
       \normalsize
       \emph{Compiled on \Datea\today} \\

       \vfill
        

	E-mail: hatam.barma@gmail.com
   \end{center}
\end{titlepage}


\tableofcontents

\clearpage
\section{Induction}
\vspace{0.5cm}


\subsection{Iteration}
\begin{itemize}
\item[-] Ordinal definition gives $x_{n}$ in terms of $n$
\item[-] Iterative definition gives $x_{n}$ in terms of $x_{n-1}$
\item[-]Terms to describe sequences \begin{itemize} \item[-] Converging \item[-] Diverging \item[-] Periodic \item[-] Chaotic \item[-] Fixed \item[-] Oscillatory \item[-] Monotonic \end{itemize}
\end{itemize}
\small
\begin{centering}
\begin{tblr}{|[.75pt]|c|c|c|c||[.75pt]}
\hline[1pt]
Type of sequence & Example & Ordinal definition & Cardinal definition \\ \hline[.75pt]
\SetCell[r=2]{c}Arithmetic & $a,\,a+d,\,a+2d,$& \SetCell[r=2]{c}$x_{n}=d(n-1)+a$ & $x_{n+1}=x_{n}+d$ \\
& $\ldots,\,a+nd$ &  & $x_{1}=a$ \\ \hline
\SetCell[r=2]{c}Geometric & $a,\,ar,\,ar^{2},$& \SetCell[r=2]{c}$x_{n}=ar^{n-1}$ & $x_{n+1}=r\cdot x_{n}$ \\
& $\ldots,\,ar^{n}$&  & $x_{1}=a$ \\ \hline[.75pt]
\end{tblr}
\end{centering}
\normalsize
\vspace{0.5cm}


\subsection{Mathematical induction}
\begin{itemize}
\item A Level FM AS / Year 1 \hspace{1cm} Pages 168 -- 169
\item A Level FM Year 2 \hspace{1cm} \phantom{AS /} Pages 2 -- 10
\end{itemize} \par
The idea behind an induction proof is to prove that a certain parameter is true for all integers $n$ by first assuming a statement is true for any arbitrary value of $n$, i.e. $n=k$, and then proving it for the next value of $n$, i.e. $n=k+1$. Then, by verifying it is true for the lowest value of $n$, i.e. $n=1$, we can deduce it is true for \emph{all} positive integer values of $n$. For example; \newline \par
Given the sequence with a cardinal definition of
\begin{equation*}
\begin{cases}
u_{n+1}&=2u_{n}+1\\
u_{n}&=1\\
\end{cases}
\end{equation*}
We claim that the ordinal definition is
\begin{equation*}
u_{n}=2^{n}-1
\end{equation*} \par
Base Case: In the case $n=1$, $u_{1}=2^{1}-1=1$, which is the correct value for $u_{1}$, so the claim holds true for the base case. \newline \par
Induction step: If the claim is true for the case $n=k$, that is $u_{k}=2^{k}-1$, then using the cardinal definition, for the case $n=k+1$
\begin{equation*}
u_{k+1}=2\left[ 2^{k}-1 \right]+1=2\times2^{k}-2+1=2^{k+1}-1
\end{equation*}
Which is exactly the claim in the case $n=k+1$. \newline \par
So, if the claim is true in the case $n=k$, then it is also true in the case $n=k+1$. However, since we have proved the claim holds true for the base case, that is $n=1$, by mathematical induction, we can prove that the claim holds true for all positive integers $n$.

\begin{itemize}
\item[Note:] Wording and language is a key part of an answer. Just writing out the induction step is insufficient for a full and proper proof, and will \emph{not} get full credit in an exam.
\end{itemize}

\vspace{0.5cm}


\subsection{Induction in other contexts}
\begin{itemize}
\item A Level FM AS / Year 1 \hspace{1cm} Pages 169 -- 176
\end{itemize} \par
Proof by induction can also be used in contexts other than proving the $n^{th}$ term of a sequence, including divisibility and inequalities.
\subsubsection*{Divisibility}
\noindent\emph{Claim:} \newline \par
$4^{n}+5^{n}+6^{n}$ is divisibile by 15 for all odd positive integers $n$. \newline \par
\noindent\emph{Proof:} \newline \par
Base case: When $n=1$, the claim gives $4^{1}+5^{1}+6^{1}=15$ which is divisible by $15$, so the claim is true in the case $n=1$. \newline \par

Inductive step: If the claim is true in the case $n=k$, that is $4^{k}+5^{k}+6^{k}$ is divisible by 15, then in the case $n=k+2$, \par
\begin{align*}
4^{k+2}+5^{k+2}+6^{k+2}&=4^{2}\cdot4^{k}+5^{2}\cdot5^{k}+6^{2}\cdot6^{k} \\
&=4^{2}\cdot4^{k}+\left( 4^{2}+9 \right)\cdot5^{k}+\left( 4^{2}+20 \right)\cdot6^{k} \\
&=4^{k}\left( 4^{k}+5^{k}+6^{k} \right)+9\cdot5^{k}+20\cdot6^{k}
\end{align*}
We already know $4^{k}+5^{k}+6^{k}$ is divisible by 15 by hypothesis, $9\cdot5^{k}$ has a factor of $3$ and $5$, and by extension $15$, as does $20\cdot6^{k}$. \newline \par

So if the claim is true in the case $n=k$, then it is also true in the case $n=k+2$. However, since the claim is true in the case $n=1$, then by mathematical induction, this is true for all odd integers $n\geq1$

\subsubsection*{Inequality}
\noindent\emph{Claim:} \newline \par
$n!>3^{n}$ for all positive integers $n\geq7$.\newline\par
\noindent\emph{Proof:}\newline\par
Base case: When $n=7$, $7!=5040$, and $3^{7}=2187$, so $7!>3^{7}$. \newline \par

Inductive step: If the claim is true in the case $n=k$, that is $k!>3^{k}$, then in the case $n=k+1$,
\begin{align*}
(k+1)!&=(k+1)(k!) & 3^{k+1}&=3\cdot3^{k}.
\end{align*}
By hypothesis, since $k!>3^{k}$, and $k+1>3$ since $k\geq7$,
\begin{equation*}
(k+1)k!>(k+1)3>3\cdot3^{k}
\end{equation*}
Therefore $(k+1)!>3^{k+1}$. \newline \par

So if the claim is true in the case $n=k$, then it is also true in the case $n=k+1$. However, since it is true for the base case where $n=7$, the claim is true for all $n\geq7$ for $n\in\mathbb{Z}$.

\subsubsection*{Matrices}
\noindent\emph{Claim:} \newline \par
For the matrix $\boldsymbol{A}=\begin{bmatrix}1&1\\0&1\end{bmatrix}$, $\boldsymbol{A}^{n}=\begin{bmatrix}1&n\\0&1\end{bmatrix}$.\newline\par
\noindent\emph{Proof:}\newline\par
Base case: When $n=1$, $\boldsymbol{A}^{n}=\begin{bmatrix}1&n\\0&1\end{bmatrix}=\begin{bmatrix}1&1\\0&1\end{bmatrix}$ so the claim holds true for the base case of $n=1$. \newline \par

Inductive step: If the claim is true in the case $n=k$, that is $\boldsymbol{A}^{k}=\begin{bmatrix}1&k\\0&1\end{bmatrix}$, then in the case $n=k+1$,
\begin{align*}
\boldsymbol{A}^{k+1}&=\boldsymbol{A}\boldsymbol{A}^{k} \\
&=\begin{bmatrix}1&1\\0&1\end{bmatrix}\begin{bmatrix}1&k\\0&1\end{bmatrix} \\
&=\begin{bmatrix}1+0&k+1\\0+0&0+1\end{bmatrix}
&=\begin{bmatrix}1&k+1\\0&1\end{bmatrix}
\end{align*}
Which is exactly the claim in the case $n=k+1$ \newline \par

So if the claim is true in the case $n=k$, then it is also true in the case $n=k+1$. However, since it is true for the base case where $n=1$, the claim is true for all $n\geq1$ for $n\in\mathbb{Z}$.

\vspace{0.5cm}

\end{document}